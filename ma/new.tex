\documentclass{article}
\title{New Developments}
\author{Dan Ibanez}
\date{Jun 26, 2014}
\begin{document}
\maketitle
This document makes a note of major new algorithmic and technical
developments going from meshadapt 1.0 to 2.0.

\section{Edge Collapse Conditions}
Not sure this was or was not solid in the 1.0 code, but
the documentation (Garimella's thesis) was not perfect...
\section{Better Non-Manifold Swap}
The 1.0 code has an ad-hoc reclassification system for
swaps on non-manifold faces.
We now take a better topological approach, but is has not
been tested with non-manifold meshes yet
\section{Short Edge Removal}
Li's thesis (and version 1.0) did not use short edge removal
as much as version 2.0.
Low quality elements can be divided into two classes:
short edges and large dihedral angles.
This initial separation and the use of Jie Wan's short
edge removal technique reduces the need for exotic operations
aimed at large dihedral angles,
since only tets without short edges are given to the large
dihedral tet fixer.
\section{CavityOp}
The Cavity Operator has become a staple of MeshAdapt algorithms.
It reduces the parallel concerns of a mesh modification operator
down to what matters: the cavity definition.
The only code that does explicit communication now is the
refinement algorithm, where the global mesh is the cavity.
Version 1.0 of MeshAdapt used to have checks all over the modification
operator code for partition boundary entities, usually resulting
in an abortion of the modification.
\section{Tetrahedronization}
The greedy algorithm that finds edge diagonal directions is completely
novel and noteworthy, and is provably correct.
\section{Load Balance Weights}
Both the metric space volume integral (as opposed to counting split edges)
and the clamping logic for out-of-range size fields.
\section{APF-Supported Solution Transfer}
\section{High-order solution transfer / curving}
\section{Crawlers}
A family of algorithms for boundary layer operations in parallel
that impose no requirements on the partitioning and never migrate.
\section{Digger}
Snapping algorithm that just uses edge collapses.
\end{document}

