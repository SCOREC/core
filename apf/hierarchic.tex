\documentclass{article}
\usepackage{amsmath}
\usepackage{amssymb}
\title{Hierarchic Shape Functions}
\author{Brian Granzow, SCOREC}
\begin{document}
\maketitle

\section{Introduction}

This document serves to provide an overview of the hierarchic shape functions
provided in apf. Hierarchic shape functions are constructed in a manner so that
the basis $\mathcal{B}^{p+1}$ for the polynomial space $P^{p+1}(\Omega^e)$
over an element $\Omega^e$ is obtained by solely by the addition of new shape
functions to the basis $\mathcal{B}^p$ for the polynomial space
$P^p(\Omega^e)$. The hierarchic shape functions $N$ can be written in terms
of kernel functions $\phi_i$ and barycentric coordinates $A_i$ for elements.

\section{Legendre polynomials}

Legendre polynomials are typically derived over the interval $[-1,1]$. The
Legendre polynomials presented here are derived over the interval $[0,1]$
as it suits the parametric space for our reference tetrahedron and triangle,
as described later. These are commonly referred to as shifted Legendre
polynomials, which can be obtained by using the substitution $x \to 2x - 1$ for
standard Legendre polynomials. The first two shifted Legendre polynomials are
defined as: $L_0 = 1$ and $L_1 = 2x - 1$. The shifted Legendre polynomials
satisfy the recursion relation:
\[
\left\{ \frac{2i-1}{i} (2x-1)\right \} L_{i-1}(x) -
\left\{ \frac{i-1}{i} \right \} L_{i-2}(x) \quad i \geq 2
\]
The first few Legendre polynomials are listed below
\begin{align*}
L_0(x) &= 1 \\
L_1(x) &= 2x - 1 \\
L_2(x) &= 6x^2 - 6x + 1 \\
L_3(x) &= 20x^3 - 30x^2 + 12x - 1 \\
L_4(x) & = 70x^4 - 140x^3 + 90x^2 - 20x + 1
\end{align*}
The Legendre polynomials satisfy are orthogonal on $[0,1]$
\[
\int _0 ^1 L_i(x) L_j(x) \; \text{d}x \; =
\begin{cases}
0 &\quad i \neq j \\
1/(2i+1) &\quad i = j
\end{cases}
\]

\section{Lobatto functions}

The Lobatto functions on $[0,1]$ are integrated Legendre polynomials, and
are defined as
\begin{align*}
l_0(x) &= 1-x \\
l_1(x) &= x \\
l_i(x) & = \frac{1}{ \| L_{i-1} \|_2}
\int _0 ^x L_{i-1}(t) \; \text{d}t \quad i \geq 2
\end{align*}
From the orthogonal property of the Legendre polynomials, it is clear that
\[
\frac{1}{ \| L_{i-1} \|_2} = \sqrt{2i - 1}
\]
The first few Lobatto functions are listed below
\begin{align*}
l_0(x) &= 1-x \\
l_1(x) &= x \\
l_2(x) &= \sqrt{3} (x^2 - x) \\
l_3(x) &= \sqrt{5} (2x^3 - 3x^2 + x) \\
l_4(x) &= \sqrt{7} (5x^4 - 10x^3 + 6x^2 - x)
\end{align*}

\section{Kernel functions}

To define shape functions, it will be useful to define kernel functions
$\phi_i$ by decomposing the Lobatto functions as follows:
\[
l_i(x) = l_0(x) l_1(x) \phi_{i-2}(x) \quad i \geq 2
\]
The first few kernel functions are listed below
\begin{align*}
\phi_0(x) &= -\sqrt{3} \\
\phi_1(x) &= -\sqrt{5} (2x - 1) \\
\phi_2(x) &= -\sqrt{7} (5x^2 - 5x + 1)
\end{align*}

\section{Reference Tetrahedron}

We define our reference tetrahedron $\Omega^e_t$ in a parametric space
such that
\[
\Omega^e_t = \left\{ \xi \in \mathbb{R}^3 \; : \;
0 < \xi_1, \xi_2, \xi_3 < 1; \; \xi_1 + \xi_1 + \xi_3 < 1 \right\}
\]
The area coordinates for the reference tetrahedron can be defined in terms of
the $\xi$ coordinates as
\begin{align*}
A_1 &= 1 - \xi_1 - \xi_2 - \xi_3 \\
A_2 &= \xi_1 \\
A_3 &= \xi_2 \\
A_4 &= \xi_3 \\
\end{align*}
There are 4 vertices associated with our parent tetrahedron. These can be
expressed in parametric space as
\begin{align*}
v_1(\xi) &= (0,0,0) \\
v_2(\xi) &= (1,0,0) \\
v_3(\xi) &= (0,1,0) \\
v_4(\xi) &= (0,0,1)
\end{align*}
They can similarly be expressed in area coordinates as
\begin{align*}
v_1(A) &= (1,0,0,0) \\
v_2(A) &= (0,1,0,0) \\
v_3(A) &= (0,0,1,0) \\
v_4(A) &= (0,0,0,1)
\end{align*}

\subsection{Vertex shape functions}

There are always only 4 shape functions associated with tetrahedral vertices, 
$v_1, v_2$, $v_3$, and $v_4$.
These are
\begin{align*}
N^{v_1}(A) &= A_1 \\
N^{v_2}(A) &= A_2 \\
N^{v_3}(A) &= A_3 \\
N^{v_4}(A) &= A_4
\end{align*}
These by themselves would coincide exactly with linear Lagrange shape
functions and are of polynomial order $p=1$.

\subsection{Edge shape functions}
Edge shape functions will be defined in terms of the shape functions
associated with the two vertices $v_i,$ and $v_j$ that bound the edge and the
kernel functions defined previously. Edge shape functions of polyonmial order
$p$ are defined by
\[
N^{e_{ij}}_p = N^{v_i} N^{v_j} \phi_{p-2}(N^{v_j} - N^{v_i})
\quad p \geq 2
\]

\section{Future work}
Currently only linear and quadratic hierarchic shape functions are
implemented for tetrahedra. This document will be updated to explain face
and region bubble modes when they are implemented for various element types.

\end{document}
